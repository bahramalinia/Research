\documentclass[11pt]{article}

\usepackage[paper=a4paper,dvips,top=4cm,left=2.5cm,right=2.5cm,
foot=2cm,bottom=4cm]{geometry}
%\usepackage{float}
\usepackage[linesnumbered,ruled,vlined]{algorithm2e}
%\usepackage{algorithm}
%\usepackage{algorithmic}
\usepackage{amssymb,amsmath}
\usepackage{caption}
\usepackage{subcaption}
\usepackage{comment}
\usepackage{color,soul}
\usepackage{ragged2e} 
\usepackage[usenames,dvipsnames]{xcolor}
%\usepackage{subfigure}
%\usepackage{flushend}

% *** CITATION PACKAGES ***
%
\usepackage{cite}
\usepackage[pdftex]{graphicx}
\graphicspath{{figRes/}{../}}

\usepackage[caption=false,font=footnotesize]{subfig}
%
% correct bad hyphenation here
\hyphenation{op-tical net-works semi-conduc-tor}
\setlength{\intextsep}{-1ex} % remove extra space above and below in-line float

\newcommand{\sa}{\textsc{SmartAllocate}}
\newcommand{\rc}{\textsc{ReConsider}}
\newcommand{\bc}{\textsc{BetaCover}}
\newcommand{\ics}{\textsc{iCS}\xspace}
\newcommand{\fcs}{\textsc{fCS}\xspace}
\newcommand{\MCSP}{\textsf{SPAN}\xspace}
\newcommand{\dMCSP}{\textsf{dSPAN}\xspace}
\newcommand{\focs}{\textsc{foCS}\xspace}
\newcommand{\iocs}{\textsc{ioCS}\xspace}
\newcommand{\finalalg}{\textsc{ioCS+}\xspace}
\newcommand{\opt}{\textsc{OPT}\xspace}
\newcommand{\alg}{\textsc{ALG}\xspace}
\newcommand{\rtl}{\textsc{GreedyRTL}\xspace}
\newcommand{\iolp}{\textsc{ioLP}\xspace}
\newcommand{\folp}{\textsc{foLP}\xspace}

\ifodd 1
\newcommand{\rev}[1]{{\color{black}#1}}%revise of the text
\newcommand{\com}[1]{\textbf{\color{red}(Bahram says: #1)}}%comment of the text
\newcommand{\comm}[1]{\textbf{\color{red}(Mohammad says: #1)}}%comment of the text
\else
\newcommand{\rev}[1]{#1}
\newcommand{\com}[1]{}
\fi

\usepackage[colorinlistoftodos, textwidth=4cm, shadow]{todonotes}
\newcommand{\bahram}[1]{\todo[inline,color=orange!40]{{\it Bahram:~}#1}}
\newcommand{\enrique}[1]{\todo[inline,color=blue!15]{#1}}
\begin{document}


\title{Response Letter for manuscript TSG-01885-2017 ``Online EV Scheduling Algorithms for Adaptive Charging Networks with Global Peak Constraints'' \\
	\vspace{4mm} \large
	by  Bahram~Alinia, Mohammad~H.~Hajiesmaili, Zachary J. Lee, Noel Crespi, and Enrique Mallada
}

\maketitle

\textbf{Dear Editor,}

We are very grateful for handling our paper and the time and effort that you and your team put into the process. We have thoroughly revised our manuscript based on your and the referees' constructive comments. This document provides a summary of changes made in the revised manuscript and detailed responses to the comments of the reviewers. The major changes in the revised manuscript are summarized below:

\begin{enumerate}
\item \textit{Better articulation of the paper's positioning, contribution, and significance}. 
\begin{itemize}
\item To highlight the positioning of our work among the extensive literature in EV charging scheduling, we added further explanations in Introduction to determine the exact scenario of interest among others. While several related work consider  EV scheduling in low-load regimes, our work tackle the resource-limited EV scheduling in high-load regimes. More details are given in response to Comment 3.1. 
\item We added a new Related Work Section in the revision (Section II) to highlight the positioning, contribution, and difference of our work as compared to the existing literature. The organization of the related work is based on the constructive Comment 1.0.3 of Reviewer 1. More details are given in response to Comments 1.0.1 and 1.0.3.
\item We re-organized and improved our the motivation statement and unique challenges of tackling ``global peak constraints'' in Introduction by considering Caltech ACN as a real example, \rev{clarifying the technical differences between scheduling problem with and without the global peak constraint}.  More details are given in response to Comment 2.2.
\end{itemize}
\item \textit{Better presentation and making the paper to be self-contained.} We have failed in our previous version of the paper to present our work in a self-contained manner. In the revision we compiled a 10-page manuscript by adding two new sections on Related Work (Section II) and offline algorithm design (Section IV-A), to the main manuscript. As mentioned in the previous item, related work helps to clarify the positioning, contribution, and significance of our work. In addition, with offline algorithm design in the main body, we do not refer to any algorithm in Appendix anymore, and it helps to follow the other online and offline algorithms easier. More details are given in responses to Comments 1.0.2 and 2.1. 

\item \textit{More clarification on the valuation model}. Since we tackle EV charging scheduling in resource-limited scenarios with peak-constraints, we formulated a \rev{social welfare maximization problem} in which the goal is to achieve optimal revenue from the heterogeneous EVs with different valuations. We clarified the notion of valuation in system model in Section III-A in Page 3. More details are given in responses to Comments 1.2, 1.4, 2.3, and 3.1.

\item \textit{Re-implementation of all simulations and providing more explanations on simulation scenarios and setup.}  The EV battery properties including charging rates, etc., that we used in the first submission were not up-to-date according to the latest battery \rev{and chargers} technologies. In the revised version, we updated those properties and re-implemented all the simulations. We explained the simulation set-up in more details in Section VI. We combined two figures to save more space and provide more room and to be able to compare our proposed online and approximate algorithms to the optimal solution and existing algorithms in single plots. More details are given in responses to Comments 1.13, 1.14, 1.15, 1.16, 2.6, and 3.2.


\end{enumerate}


In what follows, we mention first the comments (as appeared in the decision letter, highlighted in {\color{blue} blue} in this letter) followed by a description on how we addressed those comments in the paper. Once again, thank you and your team for providing very constructive comments to help in improving this paper.



%%%%%%%%%%%%%%%%%%%%%%%%%%%%%%%%%%%%%%%%%%%%%%%%%%%%%%%%%%%%%%%%%%%%%%%%%%%%%%
\newpage

%\renewcommand{\thesection}{\arabic{count})}
%\newcounter{count}
%\stepcounter{count}
%\addtocounter{section}{1}
%\setcounter{section}{1}
{\Large\textbf{Editor's Comments:}}
\vspace{3mm}

{\color{blue}All three reviewers have expressed concern regarding the necessity of appendix, which makes the paper over the page limit set by this journal. The authors are advised to reduce the length of their paper. In addition, organization of the paper needs to be improved and contribution needs to be clarified.}

\vspace{5mm}
\noindent\textbf{Response:}
Based on the constructive comments, we revised the paper in 10 pages  by adding important contents such as literature review and an important technical contribution on offline algorithm design for fractional business model. 
By adding these contents and revising the paper carefully, the contribution of our paper is highlighted by demonstrating (1) the importance, uniqueness, and challenges of our specific problem motivated by observations from Caltech adaptive charging network; and (2) differences with the similar existing works in terms of system model, and proposed technical solutions.  More details are given in the aforementioned major comments and also the specific responses to the reviewers. 


%%%%%%%%%%%%%%%%%%%%%%%%%%%%%%%%%%%%%%%%%%%%%%%%%%%%%%%%%%%%%%%%%%%%%%%%%%%%%%
%%%%%%%%%%%%%%%%%%%%%%%%%%%%%%%%%%%%%%%%%%%%%%%%%%%%%%%%%%%%%%%%%%%%%%%%%%%%%%
%%%%%%%%%%%%%%%%%%%%%%%%%%%%%%%%%%%%%%%%%%%%%%%%%%%%%%%%%%%%%%%%%%%%%%%%%%%%%%
%%%%%%%%%%%%%%%%%%%%%%%%%%%%%%%%%%%%%%%%%%%%%%%%%%%%%%%%%%%%%%%%%%%%%%%%%%%%%%
%%%%%%%%%%%%%%%%%%%%%%%%%%%%%%%%%%%%%%%%%%%%%%%%%%%%%%%%%%%%%%%%%%%%%%%%%%%%%%
%%%%%%%%%%%%%%%%%%%%%%%%%%%%%%%%%%%%%%%%%%%%%%%%%%%%%%%%%%%%%%%%%%%%%%%%%%%%%%
%%%%%%%%%%%%%%%%%%%%%%%%%%%%%%%%%%%%%%%%%%%%%%%%%%%%%%%%%%%%%%%%%%%%%%%%%%%%%%
%%%%%%%%%%%%%%%%%%%%%%%%%%%%%%%%%%%%%%%%%%%%%%%%%%%%%%%%%%%%%%%%%%%%%%%%%%%%%%
%%%%%%%%%%%%%%%%%%%%%%%%%%%%%%%%%%%%%%%%%%%%%%%%%%%%%%%%%%%%%%%%%%%%%%%%%%%%%%
%%%%%%%%%%%%%%%%%%%%%%%%%%%%%%%%%%%%%%%%%%%%%%%%%%%%%%%%%%%%%%%%%%%%%%%%%%%%%%
\newpage
\section{Reviewer $\# 1$}
{\color{blue}}
%
%\vspace{4mm}
%{\color{blue}\noindent\\
%We sincerely appreciate you for taking the time to review
%our paper. Following the concerns and suggestions, the paper
%has carefully been revised to address the reviewer's comments
%properly.
%}

{\color{blue}The authors in this paper present two online algorithms to schedule EV charging for two different business models.}
\vspace{3mm}

$\vartriangleright$ \noindent\textbf{Response:} 

We appreciate the reviewer's effort for his/her in-depth review. Below is our itemized response to this comment. 

\vspace{3mm}
{\color{blue}
\textbf{1.0.1.} While the authors try to present extensive proof on the optimality of their algorithms, the problem of EV scheduling is an old problem and the problem authors discuss are no longer the most essential/important issues. 
 }
\vspace{3mm}

	$\vartriangleright$ \noindent\textbf{Response:} 
	
We agree with the reviewer that the general EV scheduling problem has been studied extensively in the recent years, which demonstrates its importance in facilitating the deployment of EVs in energy systems. However, based on a real-world hierarchical architecture in the Adaptive Charging Network (ACN) in California Institute of Technology, in this work, we identified that the global peak constraint of the ACN results in a unique, yet challenging EV scheduling scenario that has not been addressed in the existing work.  Consequently, we formulated the corresponding EV scheduling problem and proposed solutions in two possible business models. Our experimental results also show that our scheduling mechanisms can achieve better performance as the existing scheduling practices in Caltech ACN, and at the same time the solutions can provide sound analytical performance guarantees. We further the positioning and significance of our paper in the highlighted parts of the Introduction.
	
	\vspace{3mm}
{\color{blue}	
\textbf{1.0.2.} The writing of the paper is quite confusing and hard to follow. It's more like a technical report that lacks good context. The authors simply cite certain algorithm from other paper without explaining it, making it very difficult for the reader to follow.}

\vspace{3mm}
	$\vartriangleright$ \noindent\textbf{Response:} 
	
	This is because of the page limit in the first submission. 
	In the revised manuscript, we complied a 10-pages manuscript by reorganizing the paper and adding several important contents to the main body of the paper to make it self-contained. In particular, we added an optimal offline algorithm from our technical report to the revised manuscript with detailed explanations. Hence, in the revised version there is no more algorithm cited from other references or our technical report. The details of this algorithm is in Section IV-A, Pages 4-5 of the revised manuscript. 

%\bibliography{ref-response}{}
%\vspace{-3mm}
%\bibliographystyle{ieeetr}

\end{document}